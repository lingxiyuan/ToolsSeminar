% !TeX encoding = UTF-8
% !TeX program = LuaLaTeX
% !TeX spellcheck = en_US

% Author : Yixuan Wang
% Description : Outline: Seminar on Selected Tools Week 2 --- NumPy

\documentclass[english, nochinese]{../TeXTemplate/pkuslide}

\usepackage{../TeXTemplate/def}

\title{Outline: NumPy}
\subtitle{Seminar on Selected Tools Week 2 --- NumPy}

\author{Yixuan Wang}
\date{\today}

\subject{Outline: Seminar on Selected Tools Week 2 --- NumPy}
\keywords{Python, NumPy}

\begin{document}

\begin{frame}
\titlepage
\end{frame}

\begin{frame}
\tableofcontents[subsectionstyle=hide]
\end{frame}

\section{Introduction}

\begin{frame}
\sectionpage
\end{frame}



\begin{frame}{Introduction}
\begin{enumerate}
\item NumPy’s main object is the homogeneous multidimensional array
\item Easy and fast to compute
\item Lots of useful functions included 
\item Indexing, slicing and iterating functions the same way as in Python 
\item Shape Manipulation 
\end{enumerate}
\end{frame}

\section{Data structures}

\begin{frame}
\sectionpage
\end{frame}

\begin{frame}{Data structures}
Basically there are many array types. 

Quote ``NumPy supports a much greater variety of numerical types than Python does."

Operations are similar to that in Python.
\end{frame}


\section{Basic operations}

\begin{frame}
\sectionpage
\end{frame}

\begin{frame}{Basic operations}
\begin{enumerate}
\item Broadcasting
\item Fancy indexing
\item Linear Algebra 
\item Histogram
\end{enumerate}
\end{frame}




\section{Free discussion}

\begin{frame}
\sectionpage
\end{frame}



\end{document}
